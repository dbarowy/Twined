\documentclass[10pt]{article}

% Lines beginning with the percent sign are comments
% This file has been commented to help you understand more about LaTeX

% DO NOT EDIT THE LINES BETWEEN THE TWO LONG HORIZONTAL LINES

%---------------------------------------------------------------------------------------------------------

% Packages add extra functionality.
\usepackage{times,graphicx,epstopdf,fancyhdr,amsfonts,amsthm,amsmath,algorithm,algorithmic,xspace,hyperref}
\usepackage[left=1in,top=1in,right=1in,bottom=1in]{geometry}
\usepackage{sect sty}	%For centering section headings
\usepackage{enumerate}	%Allows more labeling options for enumerate environments 
\usepackage{epsfig}
\usepackage[space]{grffile}
\usepackage{booktabs}
\usepackage{forest}
\usepackage{enumitem}   
\usepackage{fancyvrb}
\usepackage{todonotes}
\usepackage{setspace}
\usepackage{multirow}


% This will set LaTeX to look for figures in the same directory as the .tex file
\graphicspath{.} % The dot means current directory.

\pagestyle{fancy}

\lhead{Final Project}
\rhead{\today}
\lfoot{CSCI 334: Principles of Programming Languages}
\cfoot{\thepage}
\rfoot{Spring 2024}

% Some commands for changing header and footer format
\renewcommand{\headrulewidth}{0.4pt}
\renewcommand{\headwidth}{\textwidth}
\renewcommand{\footrulewidth}{0.4pt}

% These let you use common environments
\newtheorem{claim}{Claim}
\newtheorem{definition}{Definition}
\newtheorem{theorem}{Theorem}
\newtheorem{lemma}{Lemma}
\newtheorem{observation}{Observation}
\newtheorem{question}{Question}

\setlength{\parindent}{0cm}

%---------------------------------------------------------------------------------------------------------

% DON'T CHANGE ANYTHING ABOVE HERE

% Edit below as instructed

\title{Twined Language Specification} % Replace SnappyLanguageName with your project's name

\author{Lucas Weissman \and Zach Sturdevant} % Replace these with real partner names.

\begin{document}
  
\maketitle

\subsection*{Introduction}

  \begin{spacing}{1.2}
    \hspace*{1cm} "Twined" is a knowledge programming language that aims to revolutionize the way
    we manage and interact with textual information. By converting unstructured text
    data into structured, navigable knowledge graphs, "Twined" enhances comprehension
    and analysis, providing a unique interactive interface that allows users to 
    explore information and create insights through based reasoning visualizations.
    \end{spacing}
  \singlespacing

  \begin{spacing}{1.2}
  "Twined" addresses the problem of inefficient management and interaction with the 
  ever-growing volume of new information one can retain, making it a powerful tool
  for researchers, analysts, students, and anyone who works with text-based 
  information and wants to uncover hidden connections, identify patterns, and
  derive meaningful conclusions. By providing a platform that encourages 
  exploration, critical thinking, and the synthesis of ideas from various fields, 
  "Twined" aims to replicate the transformative learning experience offered by 
  liberal arts institutions by promoting users to twine the
  threads of information into their own personal tapestry of knowledge.
  \end{spacing}

\subsection*{Design Principles}

  \begin{spacing}{1.2}
    \hspace*{1cm} The design of "Twined" is guided by both aesthetic and technical ideas that aim
    to make knowledge more accessible and manageable for users. From an aesthetic
    standpoint, the language strives to provide a simple and intuitive starting
    point that gradually evolves into a powerful graphical user interface. This
    approach ensures that users of all skill levels can easily interact with 
    the language and leverage its capabilities without being overwhelmed by
    complexity. By prioritizing user experience and usability, "Twined" aims to
    create an engaging and visually appealing environment that encourages exploration
    and discovery. 
  \end{spacing}

  \begin{spacing}{1.2}
  From a technical perspective, "Twined" uses the power of generative AI to
  enhance the efficiency and effectiveness of knowledge access and management.
  The language aims to minimize user input while maximizing the usefulness
  of the output, allowing users to obtain valuable insights and connections
  with minimal effort. As the language evolves, it will incorporate features 
  that enable users to upload text and express their desired outcomes
  in natural language. "Twined" will then handle customized outputs
  to the user.
  \end{spacing}

\newpage
  \subsection*{Examples}
    \begin{enumerate}
      \item 
        \begin{enumerate}
          \item Program 1 input
            \begin{verbatim}
              {Nancy, (Fred, Sally, Sarah,)}
              {Fred, (Nancy, Sally, Sarah,)}
              {Sally, (Jeff, Jonny, Timmy,)}
              {Jeff, (Sally, Jonny, Timmy,)}
            \end{verbatim}
          \item Program 1 output
            \begin{figure}[!hb]
              \centering
              \includegraphics[width=0.3\linewidth]{images/family_tree.png} 
              \caption{Family Relationship}
              \label{figure 1:}
            \end{figure}
          \item
            This graph represents the relationships between family individuals. Each node represents 
            a person, and each edge represents a direct relationship between two individuals. For example,
            "Nancy" has relationships with "Sally," "Fred," and "Sarah," as depicted by the edges 
            connecting their respective nodes.
        \end{enumerate}
      
      \item
        \begin{enumerate}
          \item Program 2 input
            \begin{verbatim}
              {Sunlight, (PlantGrowth,)}
              {Water, (PlantGrowth,)}
              {SoilNutrients, (PlantGrowth,)}
            \end{verbatim}
          \item Program 2 output
            \begin{figure}[!hb]
              \centering
              \includegraphics[width=0.4\linewidth]{images/plant_test_0.1.png} 
              \caption{Plant Growth Relationship}
              \label{figure 2:}
            \end{figure}
          \item
            The graph represents factors affecting plant growth. Each node represents
            a factor such as "Sunlight," "Water," and "Soil Nutrients," and each edge 
            represents the influence of these factors on "Plant Growth," as depicted 
            by the edges connecting their respective nodes.
        \end{enumerate}
        
      \item
        \begin{enumerate}
          \item Program 3 input
            \begin{verbatim}
              {European Fascism, (Germany, Italy, Spain,)}
              {Germany, (Adolf Hitler,)}
              {Italy, (Benito Mussolini,)}
              {Spain,(Francisco Franco,)}
            \end{verbatim}
          \item Program 3 output
            \begin{figure}[!hb]
              \centering
              \includegraphics[width=0.5\linewidth]{images/Fascism.png} 
              \caption{Famious Eureopean Fascists}
              \label{figure 3:}
            \end{figure}
          \item
            This graph shows connections between European Fascism and leaders 
            of different Fascist countries. Each node represents a topic and 
            how those topics are connected in a top down manner.
            
        \end{enumerate}
    \end{enumerate}


\newpage
\subsection*{Language Concepts}
    \begin{spacing}{1.2}
      \hspace*{1cm} To effectively use "Twined," users should understand the fundamental concepts of
      knowledge graphs and how they represent information. While the language aims to
      minimize the need for users to grasp low-level primitives, a basic understanding
      of nodes and edges is still beneficial. Nodes represent entities or concepts 
      within the text, while edges signify the relationships or connections between 
      them. These concepts enables "Twined" to create dynamic interactive knowledge graphs 
      that highlight both hierarchical and associative relationships. By comprehending 
      how these elements interact, users can better navigate and interpret the visual 
      representation of their data. 

      \singlespacing
      "Twined" will provide an intuitive and user-friendly experience through an interface,
      for inputting their desired information to explore knowledge graphs. The user interface 
      will offer tools for navigating the graph, filtering information, and discovering new
      connections.
    \end{spacing}

  \subsection*{Formal Syntax}
    \singlespacing
      \begin{tabular}{lll}
          $<\textbf{expression}>$ & ::= & $<\textbf{Node}>$  \\
                                  & $|$  & $<\textbf{NodeInfo}>$ \\
                                  & $|$  & $<\textbf{EdgeList}>$ \\
                                  & $|$  & $<\textbf{NodeName}>$ \\
                                  & $|$  & $<\textbf{NodeList}>$ \\
                                  & $|$  & $<\textbf{Assignment}>$ \\
                                  & $|$  & $<\textbf{AssignmentList}>$ \\
                                  & $|$  & $<\textbf{NodesAndAssignments}>$ \\
                                  
          $<\textbf{Node}>$   & ::= & \{ $<\textbf{String}>$, $<\textbf{NodeName}>$ \} \\
          $<\textbf{NodeName}>$& ::= &  $<\textbf{String}>$ \\
          $<\textbf{EdgeList}>$& ::= & ($<\textbf{NodeName}>$ +) \\
          $<\textbf{NodeInfo}>$& ::= & $<\textbf{String}>$ \\
          $<\textbf{NodeLists}>$& ::= & $<\textbf{Node}>$ + \\
          $<\textbf{Assignment}>$   & ::= & \{ $<\textbf{NodeName}>$, $<\textbf{NodeInfo}>$ \} \\
          $<\textbf{AssignmentList}>$& ::= & $<\textbf{Assignment}>$ + \\
          $<\textbf{NodesAndAssignments}>$   & ::= & \{ $<\textbf{NodeList}>$, $<\textbf{AssignmentList}>$ \} \\
          
          \end{tabular}

\subsection*{Semantics}

  \begin{table}[h!]
    \begin{center}
      %\caption{Multirow table.}
      \label{tab:table1}
      \begin{tabular}{|l|c|r|l|}
        \hline
        \textbf{Syntax} & \textbf{Abstract Syntax} & \textbf{Prec./Assoc.} & \textbf{Meaning}\\
        \hline
        \text{NodeName} & Name of string & N/A & NodeName is a primitive\\
        & & & represented using an F\# string\\
        
        \hline
        \text{(name1, name2,)} & EdgeList list & N/A & An EdgeList is a list of\\
        & & & NodeNames stored in an F\# list\\
    
        \hline
        \text{NodeInfo} & Info of string & N/A & NodeInfo is a primitive\\
        & & & represented using an F\# string\\
    
        \hline
        \text{name, (name1,)} & Node of (NodeName * EdgeList) & N/A & A node is a combining form\\
        & & & that stores a NodeName and\\
        & & & a list of nodes that that node is\\
        & & & connected to in an F\# list\\
        
        \hline
        \text{\{node\} \{node\}} & Node list & N/A & A series of nodes separated\\
        & & & by whitespace including\\
        & & & newlines. These are used to \\
        & & & store all nodes in a graph\\
        
        \hline
        \texttt{\symbol{94}\symbol{94}NodeName := NodeInfo\symbol{94}\symbol{94}} & Assignment of NodeName := NodeInfo & N/A & A combining form that\\
        & & & uses a NodeName and its\\
        & & & associated information stored\\
        & & & in an F\# map\\
        
        \hline
        \texttt{\symbol{94}\symbol{94}NodeName := NodeInfo\symbol{94}\symbol{94}} & List of Assignments & N/A & A series of assignments\\
        & & & to be stored in an environment\\
        
        \hline
        \text{NodeList and AssignmentList} & Tuple of NodeList  & N/A & A list of nodes\\
        & and AssignmentList & & that form the graph\\
        & & & and the assignments of\\
        & & & each node\\
        \hline
      \end{tabular}

    \end{center}
  \end{table}
  \newpage
  \newpage
  \subsection*{Remaining Work}
  
    \begin{enumerate}
    
      \item \textbf{Backend Enhancements}
        \begin{enumerate}
        \item Enable PDF and other input formats for summarization and graph creation.
        \item Improve the parser for flexible inputs to prevent errors from missing punctuation.
        \item Optimize parsing and evaluation algorithms for larger datasets and implement caching.
        \item Refactor the backend for horizontal scaling and utilize cloud services for dynamic scaling.
        \end{enumerate}
        
      \item \textbf{Frontend Improvements}
        \begin{enumerate}
        \item Develop a sophisticated, intuitive interface using modern frameworks like React or Vue.js.
        \item Implement real-time feedback and enhance visualizations with interactive elements and theming options.
        \item Utilize JavaScript libraries like D3.js for more dynamic and visually appealing graphs.
        \item Provide export options for graphs in various formats.
        \end{enumerate}
        
      \item \textbf{User Experience and Security}
        \begin{enumerate}
        \item Enhance the web interface for smoother interaction and better navigation tools.
        \item Implement robust authentication systems, including OAuth and connection with student email.
        \item Integrate OCR technology for extracting text from images and PDFs.
        \item Add text-to-speech and speech-to-text features for enhanced accessibility.
        \end{enumerate}
      
      \item \textbf{Advanced Features and Documentation}
        \begin{enumerate}
        \item Integrate machine learning for insights, recommendations, and automatic node categorization.
        \item Develop thorough documentation and create a community support portal.
        \end{enumerate}

      \item \textbf{Features Not Implemented}
        We had planned to implement several additional features and commands, such as 
        direct interaction with OpenAI's interface for chatting, and the use of OCR technology
        to convert PNG/JPG images to work with notes on paper. Additionally, we attempted to
        work with PDF-to-text conversion. Unfortunately, due to time constraints, we were unable
        to fully develop and integrate these functionalities by the final submission.
  \end{enumerate}

  \newpage
\subsection*{}

\newpage
  \begin{enumerate}
    \item $[NO]$ You have created a 5-10 minute video presentation, and included this video with your implementation. Note: large video files cannot be added to git; please include a link to Google Drive/YouTube/etc. instead.
    \item $[YE]$  Your project has a (silly) name.
    \item $[NO]$ Your project has a specification.
    \item $[YE]$ Your parser is in a file called \texttt{Parser.fs}.
    \item $[YE]$ Your AST is in a file called \texttt{AST.fs}.
    \item $[YE]$ Your interpreter/evaluator is in a file called \texttt{Evaluator.fs}.
    \item $[YE]$ Your main function is in a file called \texttt{Program.fs}.
    \item $[YE]$ Your project compiles (this is very important).
    \item $[YE]$ Your project runs (this is very important).
    \item $[YE]$ Your language “does something.” It prints out a computed result, it generates a file, etc.
    \item $[???]$ You are sure to tell the user (me) what the expected result should be!
    \item $[YE]$ Your implementation has a test suite and it runs.
    \item $[YE]$ There is at least one test written for the parser.
    \item $[YE]$ There is at least one test written for the interpreter.
    \item $[???]$ Your project can be run at the project level by calling \texttt{dotnet run <input>} or at the solution level by calling \texttt{dotnet run --project <whatever.fsproj> <input>}.
    \item $[YE]$ Your specification document has a title.
    \item $[YE]$ Your name and your partner’s name is written at the top of the spec.
    \item $[YE]$ Your specification has an Introduction section consisting of 2+ paragraphs.
    \item $[YE]$ Your specification has a Design Principles section consisting of 1+ paragraphs.
    \item $[YE]$ Your specification has an Examples section consisting of 3+ short examples.
    \item $[NO]$ Each of your examples is provided (and ideally, each example is in a separate file) so that an interested third party (me or one of your classmates) can run them. Instructions to run the examples is provided.
    \item $[YE]$ Your specification has a Language Concepts section consisting of 1+ paragraphs.
    \item $[YE]$ Your specification has a Formal Syntax section consisting of as much BNF is needed to completely describe your language’s syntax.
    \item $[YE]$ Your specification has a Semantics section consisting of one short description per language element(where an element is normally an AST node), completely describing your language.
    \item $[NO]$ Your specification provides enough detail that an interested third-party (like me or one of your classmates) can write a new program in your language.
    \item $[\hspace{1em}]$ If your parser is “generous” and accepts programs that actually do not make sense, make sure that the program detects these cases and shuts down cleanly (i.e., the user does not see an exception).
    \item $[YE]$ You committed your specification, in a folder called docs, to a branch called final-submission.
    \item $[YE]$ You committed your implementation, in a folder called code, to a branch called final-submission.
    \item $[NO]$ Your specification has a Remaining Work section that describes further enhancements, if any, you would like to see. If your prior draft had features that you did not get to by the final submission, briefly explain why you were not able to implement them.
    \item $[\hspace{1em}]$ If you would like me to transfer ownership of your repository to you, please provide the name of a Github user or organization that will take over ownership. Put this username in a file called TRANSFER.txt.
    \item $[\hspace{1em}]$ If you intend to make your repository public, put a LICENSE.txt copyright statement in your repository, at the root level, so that people know under what conditions you plan to let them use your code. For example, provide a copy of the GNU Public License, or BSD License, etc.
    \item $[\hspace{1em}]$ Is it OK for me to share your project with future CSCI 334 students? Students often tell me that it is helpful to see what others did. If so, please add a file called SHARING.txt to your repository. If you have any conditions on sharing, please put your notes in the f ile, otherwise it’s OK to leave the file blank. If you do not want to share, do not create the file.
  \end{enumerate}

  \subsection*{Remaining Work}

% DO NOT DELETE ANYTHING BELOW THIS LINE
\end{document}
